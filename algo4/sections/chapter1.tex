\chapter{Introduction}

\begin{definition}[algorithm]
	un algorithme est caractérisé par ces 3 structures:
	\begin{enumerate}
		\item structure séquentielle: l'étape n ne se passe pas avant l'étape n-1
		\item structure conditionnelle: certaines sont executés seulement si une condition est résolu
		\item structure repetitive: des étapes peuvent être répétés
	\end{enumerate}
	aussi, un algorithme suit toujours ces propriétés:
	\begin{enumerate}
		\item il y a toujours une fin (finitude)
		\item il n'y a pas d'étapes aléatoires (déterminisme)
		\item il y a toujours une entrée et une sortie (attention, dans le monde réél, même si une fonction est de type void ou ne retourne generalement rien, ses modifications a l'environnement qui l'entoure est la sortie)
	\end{enumerate}
\end{definition}

un algorithme est souvent conçu par ces étapes:
\begin{enumerate}
	\item on énonce le problème et ses entrées/sorties
	\item puis, on essaye de le formaliser (cette étape est utile car elle sert aussi a trouver le format approprié pour encoder les instances)
	\item abstraction (c'est un peu une continuation de l'étape 2)
	\item puis l'algorithme est conçu a partir du format d'instance qu'on a décidé precedemment
\end{enumerate}


\begin{definition}[instance]
les données d'un problème sont appelé une instance, et par extension leur versions encodé servant d'entrée pour un certain algorithme aussi
\end{definition}
