\newcommand{\znz}[1] {
    $\mathbb{Z}/#1\mathbb{Z}$
}

\chapter{groupes et anneau \znz[n]}

\begin{definition}[groupe]
    Un groupe $G$ est un ensemble muni d'une loi de composition si et seulement si:
    \begin{itemize}
	\item La loi admet un element neutre $e$ tel que 
	\begin{equation}
	    \forall a, \exists e \in G^2, a \circ e = e \circ a = a
	\end{equation}

	\item Tout élement possède un symmetrique (noté $sym_{\circ}(a)$) tel que
	\begin{equation}
	    \forall a \in G \exists sym_{\circ}(a), a \circ sym_{\circ}(a) = e
	\end{equation}

	\item la loi est associative, càd
	\begin{equation}
	    \forall a, b, c \in G^3, (a \circ b) \circ c \equiv a \circ (b \circ c)
	\end{equation}
    \end{itemize}
    si de plus, la loi est commutative, càd $a \circ b \equiv b \circ a$ alors le group est dit commutatif
\end{definition}


\begin{definition}[L'anneau \znz[n] ]
    l'anneau \znz[n] est un ensemble contenant tout les entiers de $0$ à $i$.\\
\end{definition}



\section{Generateurs}

\begin{definition}[Generateur]
    un generateur est un nombre faisant partie de l'anneau $\mathbb{Z}/_n{\mathbb{Z}}$
\end{definition}


\begin{definition}[Trouver le symmetrique]
    il est assez facile de trouver le symmetrique additive d'un élément de \znz{n}. \\
    En effet, $sym_+(a) = n-a$, dans \znz{10} par exemple, $sym_+(5)= 10 - 5 = 5, 5+5 = 0$.
    il est plus compliqué de trouver un symmetrique multiplicative.
\end{definition}
