\chapter{Matrices}
\section{regles de notation}

\begin{definition}[Matrice]

\end{definition}

\begin{definition}[matrice identite]
la matrice identite est une matrice carre contenant seulemant des 1 dans sa diagonale, et des 0 dans tout les autres cases.
\end{definition}

\section{definitions}
\begin{definition}[transpose]
la transposition est une operation qui change les lignes en colonnes, et les colonnes en ligne.
elle est obtenue en faisant "pivoter" les elements de la matrice autour de la diagonale.
\end{definition}

\begin{definition}[matrice inverse]
	comme dans les autres espaces, "inverse" signifie que c'est la matrice symmetrique a une autre matrice par rapport a une operation, ici la multiplication.
formellement, $\text{pour une matrice M,} \exists M^{-1} t.q M \cdot M^-1 \equiv M_{Id}$. \
remarquez le $\exists M$. 
En effet, pas toute les matrices possedent un inverse, et ceux qui en possedent un sont appelees \textbf{matrices inversibles}.
\end{definition}


\begin{definition}{determinant}
le determinant d'une matrice A est donne par la formule \
$$ det(A)=\sum_{j=1}^n (-1)^{i+j} a_{ij}{\rm det}(A_{ij}) $$
ou $A_{ij}$ est la matrice obtenue en enlevant la ligne i et la colonne j de A.
Cette formule est valable quelle que soit la ligne $i$
choisie. Elle fait intervenir tous les coefficients $a_{ij}$ de cette ligne. Chacun de ces coefficients étant multiplié par le déterminant d'une matrice $(n−1) \times (n−1)$, il s'agit donc d'une définition récursive. Etant donné que cette formule donne le même résultat quelle que soit la ligne i choisie, on a intérêt à chercher dans A

la ligne possèdant le plus de coefficients nuls pour réduire le nombre de calculs. Si la matrice ne contient que des coefficients non nuls, il n'y a pas à priori de stratégie particulière à adopter.

Une formule équivalente existe en choisissant cette fois-ci une colonne j
quelconque dans la matrice A :
$$ det(A)=\sum_{i=1}^n (-1)^{i+j} a_{ij}{\rm det}(A_{ij}) $$
\end{definition}

\begin{definition}[calcul inverse de matrice]
Afin de calculer efficacement l'inverse d'une matrice A, on peut appliquer l'algorithme de Gauss de la façon suivante :

\begin{itemize}
    \item Appliquer l'algorithme de Gauss à la matrice A afin d'obtenir une matrice triangulaire supérieure $T_A$. Pour chaque transformation effectuée au cours de l'algorithme, effectuer exactement la même transformation sur la matrice identité.
A la fin de ce procédé vous avez obtenu 2 matrices :
\begin{itemize}
\item $T_A$ qui est la matrice A modifiée par l'algorithme de Gauss
\item $I_A$ qui est la matrice identité sur laquelle on a appliqué l'ensemble des transformations effectuées sur A. En particulier on a $I_A \times A=T_A$.
\end{itemize}

\item En partant de la dernière colonne de TA et en remontant jusqu'à la deuxième colonne, appliquer le principe de Gauss afin de placer des 0 dans chaque colonne, au-dessus de l'élément diagonal. Remarque : tous les éléments diagonaux de TA étant non nuls, il n'y aura dans ce cas, aucune permutation de lignes à effectuer. Pour chaque transformation effectuée au cours de ce deuxième passage, effectuer exactement la même transformation sur la matrice IA.

\item A l'issu de ce procédé on obtient donc 2 matrices :

\begin{itemize}
\item DA qui est la matrice TA
modifiée par l'algorithme de Gauss et qui est donc à présent une matrice diagonale.
\item SA qui est la matrice IA sur laquelle on a appliqué l'ensemble des transformations effectuées sur TA. En particulier on a $S_A \times A = D_A$
\end{itemize}

L'inverse de A est obtenu en divisant pour chaque ligne i
de SA les coefficients de la ligne par l'élément diagonal situé sur la ligne i de la matrice DA.
\end{itemize}
\end{definition}

