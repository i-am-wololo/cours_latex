\begin{exercise}
	Dans $\mathbb{R}^3$, donnez:
	\begin{enumerate}
		\item un exemple de famille generatrice qui n'est pas libre
		\item un exemple de famille libre non generatrice
		\item une famille ni libre ni generatrice
		\item donne un exemple de base autre que la famille $((1,0,0), (0, 1, 0), (0, 0, 1))$
	\end{enumerate}
\end{exercise}

pour numero 1, il suffit d'ajouter un vecteur qui est une combinaison lineaire d'un autre.
soit la base $((1,0,0), (0, 1, 0), (0, 0, 1))$, il suffit qu'elle devienne $((1,0,0), (0, 1, 0), (0, 0, 1), (2, 0, 0))$ pour qu'elle reste generatrice mais non lineaire$((1,0,0), (0, 1, 0), (0, 0, 1))$, \

pour numero 2, il suffit d'en enlever un. $((1,0,0), (0, 1, 0)$ est lineaire mais non generatrice dans $\mathbb{R^3}$ \

numero 3, $((0, 3, 0), (0, 1, 0))$. cette famille est ni libre, car un vecteur est le triple d'un autre, ni generatrice, car il est impossible d'obtenir les autre composantes a partir de cette famille. \

numero 4, j'ai choisi de remplacer une des composantes par -1, $((-1,0,0), (0, 1, 0), (0, 0, 1))$.
elle reste generatrice car on est dans $\mathbb{R}$, et $-1 \cdot (-1, 0, 0) \equiv 1 \cdot (1, 0, 0)$

